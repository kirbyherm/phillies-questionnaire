%%%%%%%%%%%%%%%%%%%%%%%%%%%%%%%%%%%%%%%%%
% Short Sectioned Assignment
% LaTeX Template
% Version 1.0 (5/5/12)
%
% This template has been downloaded from:
% http://www.LaTeXTemplates.com
%
% Original author:
% Frits Wenneker (http://www.howtotex.com)
%
% License:
% CC BY-NC-SA 3.0 (http://creativecommons.org/licenses/by-nc-sa/3.0/)
%
%%%%%%%%%%%%%%%%%%%%%%%%%%%%%%%%%%%%%%%%%

%----------------------------------------------------------------------------------------
%	PACKAGES AND OTHER DOCUMENT CONFIGURATIONS
%----------------------------------------------------------------------------------------

\documentclass[paper=a4, fontsize=11pt]{scrartcl} % A4 paper and 11pt font size

\usepackage[T1]{fontenc} % Use 8-bit encoding that has 256 glyphs
\usepackage[english]{babel} % English language/hyphenation
\usepackage{amsmath,amsfonts,amsthm} % Math packages
\usepackage{listings}
\usepackage{lipsum} % Used for inserting dummy 'Lorem ipsum' text into the template

\usepackage{siunitx}
\sisetup{output-exponent-marker=\ensuremath{\mathrm{e}}}

\usepackage{graphicx}
\graphicspath{{./}}

\usepackage{amsmath}
\DeclareMathOperator*{\argmax}{arg\,max}
\DeclareMathOperator*{\argmin}{arg\,min}
\DeclareMathOperator\erf{erf}

% \usepackage{sectsty} % Allows customizing section commands
% \allsectionsfont{\centering \normalfont\scshape} % Make all sections centered, the default font and small caps

% \usepackage{fancyhdr} % Custom headers and footers
% \pagestyle{fancyplain} % Makes all pages in the document conform to the custom headers and footers
% \fancyhead{} % No page header - if you want one, create it in the same way as the footers below
% \fancyfoot[L]{} % Empty left footer
% \fancyfoot[C]{} % Empty center footer
% \fancyfoot[R]{\thepage} % Page numbering for right footer
% \renewcommand{\headrulewidth}{0pt} % Remove header underlines
% \renewcommand{\footrulewidth}{0pt} % Remove footer underlines

\renewcommand{\thesubsection}{\thesection.\alph{subsection}} % Designate subsections by alphas

\setlength{\headheight}{13.6pt} % Customize the height of the header

\numberwithin{equation}{section} % Number equations within sections (i.e. 1.1, 1.2, 2.1, 2.2 instead of 1, 2, 3, 4)
\numberwithin{figure}{section} % Number figures within sections (i.e. 1.1, 1.2, 2.1, 2.2 instead of 1, 2, 3, 4)
\numberwithin{table}{section} % Number tables within sections (i.e. 1.1, 1.2, 2.1, 2.2 instead of 1, 2, 3, 4)
\numberwithin{equation}{subsection}

\setlength\parindent{0pt} % Removes all indentation from paragraphs - comment this line for an assignment with lots of text

%----------------------------------------------------------------------------------------
%	TITLE SECTION
%----------------------------------------------------------------------------------------

\newcommand{\horrule}[1]{\rule{\linewidth}{#1}} % Create horizontal rule command with 1 argument of height

\title{	
\normalfont \normalsize 
\textsc{Michigan State University} \\ [25pt] % Your university, school and/or department name(s)
\textsc{Statistics and Data Analysis - PHY950} \\ [23pt]
\horrule{0.5pt} \\[0.4cm] % Thin top horizontal rule
\huge Homework 5 \\ % The assignment title
\horrule{2pt} \\[0.5cm] % Thick bottom horizontal rule
}

\author{Kirby Hermansen} % Your name

\date{\normalsize\today} % Today's date or a custom date

\begin{document}

\maketitle % Print the title

%----------------------------------------------------------------------------------------
%	PROBLEM 1
%----------------------------------------------------------------------------------------


First thing to consider is to confirm that the player knows where he should be throwing the ball under every situation. If he is hesitating or unsure where the throw should go, that doubt may already be enough to cause the throw to miss the mark.



\numberwithin{equation}{subsection}
\section{Experimental Measurements and Fisher Information Matrices}
The relevant code for this section is found in p1.py.

\begin{align}
\begin{split}
\mathcal{I}_{ij} = \mathbb{E}\left[\sum_{m=1}^N \frac{n_m}{\sigma_m^2} \frac{\partial f_m}{\partial \theta_i}  \frac{\partial f_m}{\partial \theta_j}\right]
\end{split}
\end{align}

and the three equations

\begin{align}
\begin{split}
f_1(\sqrt{s_1}) = \alpha \sqrt{s_1} + \beta
\\f_2(\sqrt{s_2}) = \alpha \sqrt{s_2} + \beta
\\f_3(\sqrt{s_3}) = \alpha \sqrt{s_3} + \beta
\end{split}
\end{align}

we can iterate over all possible experimental configurations $n = [n_1, n_2, n_3]$ subject to our financial constraint

\begin{align}
\begin{split}
\sum_{i=1}^3 c_i n_i \leq 10
\end{split}
\end{align}

where $c = [1,2,3]$ is the cost per measurement at each CoM energy. Additionally, the experiment must be run at least once at two different energies in order to obtain estimates for both $\alpha$ and $\beta$. And lastly, running at $\sqrt{s_3} = 5 \textrm{GeV}$ requires at least one run at each of the other two energies first.

Considering all these constraints, and iterating over all remaining possibilities, yields the following minimum variances (calculated by inverting the Fisher Matrix) for $\alpha$ and $\beta$

\begin{align}
\begin{split}
\sigma_\alpha^2 = 0.1962 \quad n = [2, 1, 2] 
\\ V_\alpha = \left[\begin{matrix}
0.1962 & -0.3793
\\-0.3793 & 1.1785
\end{matrix}\right]
\end{split}
\end{align}

\begin{align}
\begin{split}
\sigma_\beta^2 = 0.7521 \quad n = [5, 1, 1] 
\\ V_\beta = \left[\begin{matrix}
0.2978 & -0.3895
\\-0.3895 & 0.7521
\end{matrix}\right]
\end{split}
\end{align}

\subsection{Incorporation of Prior Results}

\begin{align}
\begin{split}
\tilde{\mathcal{I}}_{ij} = \mathbb{E}\left[\sum_{m=1}^N \frac{n_m}{\sigma_m^2} \frac{\partial f_m}{\partial \theta_i}  \frac{\partial f_m}{\partial \theta_j}\right] + \mathcal{I}_{prior}
\end{split}
\end{align}

where 

\begin{align}
\begin{split}
\mathcal{I}_{prior} = \left[ \begin{matrix}
(0.1 \alpha)^{-2} & 0 \\ 0 & (0.2 \beta)^{-2}
\end{matrix}\right]
\end{split}
\end{align}

Using this $\tilde{\mathcal{I}}$ we can recalculate the variances for $\alpha, \beta$ and find the following

\begin{align}
\begin{split}
\tilde{\sigma}_\alpha^2 = 0.0571 \quad n = [2, 1, 2] 
\\ \tilde{V}_\alpha = \left[\begin{matrix}
0.0571 & -0.0292
\\-0.0292 & 0.1326
\end{matrix}\right]
\end{split}
\end{align}

\begin{align}
\begin{split}
\tilde{\sigma}_\beta^2 = 0.1088 \quad n = [8, 1, 0] 
\\ \tilde{V}_\beta = \left[\begin{matrix}
0.0981 & -0.0516
\\-0.0516 & 0.1088
\end{matrix}\right]
\end{split}
\end{align}

And so we see that including the prior measurements does not change which method to use for estimating the slop, but does dramatically decrease the variance of those measurements. Additionally, including the prior measurements means that to estimate the intercept we can focus on just the measurement at the first energy ($n = [8,1,0]$ now) and dramatically reduce our variance on the intercept estimate.



\begin{figure}
%\includegraphics[scale=0.8]{src/alpha-beta.png}
\caption{Effects of varying the theoretical values of $\alpha$ and $\beta$ on variances of the parameters}
\end{figure}



\numberwithin{equation}{subsection}
\section{Brownian Motion Experiment}

The relevant code for this section is found in p2.py.

\subsection{Determination of $\nu_0$, $k$ using Log-likelihood} \label{loglike}


The calculated $\nu_0$, $k$ values which maximize the log-likelihood function are determined to be 


\subsection{Determination of $N_A$ from $k$}

Using the given formula and values for $R$, and incorporating our result from Sec \ref{loglike}, $N_A$ can be easily calculated yielding:

\begin{align}
\begin{split}
{N}_{A, calc} = R/{k} = \num{6.941e+23}
\end{split}
\end{align}

This is fairly close to the accepted value of $N_{A, act}=\num{6.022e23}$.

\subsection{Determination of $\nu_0$, $k$ using $\chi^2$}

Using the same methods as in Sec \ref{loglike}, but now using the given $\chi^2$ formula we approximate values of $\nu_0$, $k$ to be:
\begin{align}
\begin{split}
\nu_0 = 1846 \quad k = \num{1.1992e-16}
\end{split}
\end{align}

with a $\chi^2$ value 

\begin{align}
\begin{split}
\chi^2_{min} = 4.5690
\end{split}
\end{align}

See Fig \ref{chicontour} for the contour plot around this minimum.

Using this $\chi^2_{min}$ value and the number of degrees of freedom, $\textrm{ndof} = n_{obs} -1 = 3$, we find the $p$-value from $1-\textrm{cdf}(\chi^2_{min}, \textrm{ndof})$

\begin{align}
\begin{split}
p = 0.7938
\end{split}
\end{align}

This tells us this is a good fit for the data, though the calculated value is not precisely the accepted value for $N_A$ or $k$.

Regarding systematic errors in the experiment, I would hazard a guess that the fact that the liquid is held in suspension in a container of fixed size could affect the particles' Brownian motion in the container. Given that the particles are restricted to motion in a given box, even taking a small sample near the center (I assume), the particles may not exhibit the random motion required for Brownian motion due to the boundary conditions. This would then bias the measurements and would not allow for as accurate a determination of $k$ and $N_A$.


%----------------------------------------------------------------------------------------

\end{document}
