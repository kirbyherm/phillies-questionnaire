%%%%%%%%%%%%%%%%%%%%%%%%%%%%%%%%%%%%%%%%%
% Short Sectioned Assignment
% LaTeX Template
% Version 1.0 (5/5/12)
%
% This template has been downloaded from:
% http://www.LaTeXTemplates.com
%
% Original author:
% Frits Wenneker (http://www.howtotex.com)
%
% License:
% CC BY-NC-SA 3.0 (http://creativecommons.org/licenses/by-nc-sa/3.0/)
%
%%%%%%%%%%%%%%%%%%%%%%%%%%%%%%%%%%%%%%%%%

%----------------------------------------------------------------------------------------
%	PACKAGES AND OTHER DOCUMENT CONFIGURATIONS
%----------------------------------------------------------------------------------------

\documentclass[paper=a4, fontsize=11pt]{scrartcl} % A4 paper and 11pt font size

\usepackage[T1]{fontenc} % Use 8-bit encoding that has 256 glyphs
\usepackage[english]{babel} % English language/hyphenation
\usepackage{amsmath,amsfonts,amsthm} % Math packages
\usepackage{listings}
\usepackage{lipsum} % Used for inserting dummy 'Lorem ipsum' text into the template

\usepackage{siunitx}
\sisetup{output-exponent-marker=\ensuremath{\mathrm{e}}}

\usepackage{graphicx}
\graphicspath{{./}}

\usepackage{amsmath}
\DeclareMathOperator*{\argmax}{arg\,max}
\DeclareMathOperator*{\argmin}{arg\,min}
\DeclareMathOperator\erf{erf}

% \usepackage{sectsty} % Allows customizing section commands
% \allsectionsfont{\centering \normalfont\scshape} % Make all sections centered, the default font and small caps

% \usepackage{fancyhdr} % Custom headers and footers
% \pagestyle{fancyplain} % Makes all pages in the document conform to the custom headers and footers
% \fancyhead{} % No page header - if you want one, create it in the same way as the footers below
% \fancyfoot[L]{} % Empty left footer
% \fancyfoot[C]{} % Empty center footer
% \fancyfoot[R]{\thepage} % Page numbering for right footer
% \renewcommand{\headrulewidth}{0pt} % Remove header underlines
% \renewcommand{\footrulewidth}{0pt} % Remove footer underlines

\renewcommand{\thesubsection}{\thesection.\alph{subsection}} % Designate subsections by alphas

\setlength{\headheight}{13.6pt} % Customize the height of the header

\numberwithin{equation}{section} % Number equations within sections (i.e. 1.1, 1.2, 2.1, 2.2 instead of 1, 2, 3, 4)
\numberwithin{figure}{section} % Number figures within sections (i.e. 1.1, 1.2, 2.1, 2.2 instead of 1, 2, 3, 4)
\numberwithin{table}{section} % Number tables within sections (i.e. 1.1, 1.2, 2.1, 2.2 instead of 1, 2, 3, 4)
\numberwithin{equation}{subsection}

\setlength\parindent{0pt} % Removes all indentation from paragraphs - comment this line for an assignment with lots of text

%----------------------------------------------------------------------------------------
%	TITLE SECTION
%----------------------------------------------------------------------------------------

\newcommand{\horrule}[1]{\rule{\linewidth}{#1}} % Create horizontal rule command with 1 argument of height

\title{	
\normalfont \normalsize 
\textsc{Michigan State University} \\ [25pt] % Your university, school and/or department name(s)
\horrule{0.5pt} \\[0.4cm] % Thin top horizontal rule
\huge Philadelphia Phillies R \& D Questionnaire \\ % The assignment title
\horrule{2pt} \\[0.5cm] % Thick bottom horizontal rule
}

\author{Kirby Hermansen} % Your name

\date{\normalsize\today} % Today's date or a custom date

\begin{document}

\maketitle % Print the title

%----------------------------------------------------------------------------------------
%	PROBLEM 1
%----------------------------------------------------------------------------------------

\numberwithin{equation}{subsection}

\section{Outfielder Proposal}

The issue at the heart of this question is unknown. Given the outfield coach's report, Player X could have any number of issues with his performance, which may or may not be treatable with drills or targeted research. Having said this, there are several potential problems to troubleshoot and help us and other members of the organization better understand the situation. Here, I will lay out my proposed troubleshooting methodology, as well as the indicative data and potential solutions. Simplified results are shown in Tab. \ref{tab:of}

\subsection{Arm Evaluation}

As the coach specifically mentions how good Player X's arm is, I get the feeling that both his accuracy and arm strength are not major issues but they are still worth addressing.

\subsubsection{Accuracy}

First thing to consider is to confirm that the player knows where he should be throwing the ball under every situation. If he is hesitating or unsure where the throw should go, that doubt may already be enough to cause the throw to miss the mark. This can be evaluated by seeing if there is any consistency to the types of plays or the particular target on the field when Player X has made a throwing error (it also would be worth it for the outfield coach to quiz Player X on a few situations to be sure). 

Possible remedies include studying the situational aspect and practicing making the target throws.

\subsubsection{Arm Strength}

Since the two concerns the coach highlights are ``errors'' and ``failing to get to some balls'' I suspect that arm strength is not a major concern for Player X. However, it is worth it to quickly confirm that he is not ``overthrowing'' on throws. Here it would be useful to check 

\subsection{Fielding Evaluation}

Given the coach's expressed concerns I would expect the main issue would be in Player X's performance fielding hit balls.

\subsubsection{Fielding Positioning} \label{subsub:fieldpos}

All hitters have a spray chart showing the outcomes of their at-bats \cite{baseballsavant}. It would be useful to compare that spray chart to the outfielder's original positioning for each case and calculate the difference. For the case of Player X, it may be that, when compared with the average of other fielders, he is actually being asked to cover greater distances to make plays, which may either be due to poor luck (in which case the discrepancy would go away under a larger sample size) or poor positioning by the coaches.  



\subsubsection{Fielding Range} \label{subsub:fieldrange}

As in Sec. \ref{subsub:fieldpos}, calculating the distance that Player X has to cover on his plays gives us an idea of his fielding range. Moreover, it is possible to highlight the plays where he is underperforming and dig deeper into what may have caused this. For instance, there may be plays where he failed to complete an out despite not having to cover as much ground. These likely will be cases where the launch angle and/or exit velocity off the bat is much more difficult to field quickly (a hard line drive) compared to a lazy fly ball. If there are in fact plays where he is not converting lazy fly balls into outs, these are incredibly important to identify and resolve. Regardless, his actions on either set of plays can give an insight into what may be causing his fielding issues.

\subsubsection{Ballpark Features}

Finally, all ballparks are inherently different, and this has particular impact on an outfielder's play. It is important to verify that this is not the case for Player X. In order to do so, I would compare the metrics disccussed in Sec \ref{subsub:fieldpos} and \ref{subsub:fieldrange} in his performance at home or away and see if there is a notable difference. If so, more practice with the features of that ballpark would be in order.

\subsection{Results}

The different metrics discussed in Sec \ref{sub:arm} and \ref{sub:field} are by no means an exhaustive list of the potential hazards which are hampering Player X's performance, but would provide a decent start at troubleshooting the problems. 

\begin{center}
\begin{tabular}{ |c|c|c| } 
 \hline
 Problem & Data to Diagnose & Solution \\ [0.5ex] 
 \hline
 \hline
 Fielding Positioning & 
 Fielding Range &
 Ballpark Features &
 Arm Accuracy & 
 Arm Strength &

 \hline
\end{tabular}
\end{center}
\section{ OBP Predictor }



\bibliographystyle{plain}
\bibliography{refs}

%----------------------------------------------------------------------------------------

\end{document}
